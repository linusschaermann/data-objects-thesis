\documentclass[12pt,oneside]{article}

%%%%%%%%%%%%%%%%%%%%%%%%%%%%
%%   Zusaetzliche Pakete  %%
%%%%%%%%%%%%%%%%%%%%%%%%%%%%
\usepackage{enumerate}  
\usepackage{fancyhdr}
\usepackage{a4wide}
\usepackage{graphicx}
\usepackage{palatino}
\usepackage{multirow}
\usepackage{booktabs}
\usepackage{titlesec}
\usepackage{acronym}% http://ctan.org/pkg/acronym
\usepackage{enumitem}% http://ctan.org/pkg/enumitem

%folgende Zeile auskommentieren für englische Arbeiten
%\usepackage[ngerman]{babel}
%folgende Zeile auskommentieren für deutsche Arbeiten
\usepackage[ngerman, english]{babel}

\usepackage[T1]{fontenc}
\usepackage[utf8]{inputenc}
\usepackage[bookmarks]{hyperref}
\usepackage[justification=centering]{caption}
\usepackage[style=apa,natbib=true,backend=biber,maxbibnames=20]{biblatex}
\usepackage{csquotes}
\bibliography{literature}

\setlength{\parindent}{0em} 
\setlist[itemize]{noitemsep, topsep=0pt}
\setlist[enumerate]{noitemsep, topsep=0pt}


%%%%%%%%%%%%%%%%%%%%%%%%%%%%%%
%% Definition der Kopfzeile %%
%%%%%%%%%%%%%%%%%%%%%%%%%%%%%%

\pagestyle{fancy}
\fancyhf{}
\cfoot{\thepage}
\setlength{\headheight}{16pt}

%%%%%%%%%%%%%%%%%%%%%%%%%%%%%%%%%%%%%%%%%%%%%%%%%%%%%
%%  Definition des Deckblattes und der Titelseite  %%
%%%%%%%%%%%%%%%%%%%%%%%%%%%%%%%%%%%%%%%%%%%%%%%%%%%%%

\newcommand{\JMUTitle}[9]{

  \thispagestyle{empty}
  \vspace*{\stretch{1}}
  {\parindent0cm
  \rule{\linewidth}{.7ex}}
  \begin{flushright}
    \vspace*{\stretch{1}}
    \sffamily\bfseries\Huge
    #1\\
    \vspace*{\stretch{1}}
    \sffamily\bfseries\large
    #2\\
    \vspace*{\stretch{1}}
    \sffamily\bfseries\small
    #3
  \end{flushright}
  \rule{\linewidth}{.7ex}

  \vspace*{\stretch{1}}
  \begin{center}
    \includegraphics[width=2in]{siegel} \\
    \vspace*{\stretch{1}}
    \Large Bachelorarbeit  \\

    \vspace*{\stretch{2}}
   \large Lehrstuhl f\"{u}r Wirtschaftsinformatik\\
    \large und Systementwicklung\\
    \large Universität Würzburg\\
    \vspace*{\stretch{1}}
    \large Betreuer:  #8 \\[1mm]
    
    \vspace*{\stretch{1}}
    \large W\"urzburg, den #7 \\
        \vspace*{\stretch{0.25}}

    % Bearbeitungszeit: 14.03.2025 - 09.05.2025 % Die Bearbeitungszeit der Seminar-/ Abschlussarbeit ist hier einzutragen.

  \end{center}
}

\titlespacing*{\section}
{0pt}{3.5ex plus 1ex minus .2ex}{.2ex plus .2ex}
\titlespacing*{\subsection}
{0pt}{1.5ex plus 1ex minus .2ex}{.2ex plus .2ex}
\titlespacing*{\subsubsection}
{0pt}{1.5ex plus 1ex minus .2ex}{.2ex plus .2ex}




%%%%%%%%%%%%%%%%%%%%%%%%%%%%
%%  Beginn des Dokuments  %%
%%%%%%%%%%%%%%%%%%%%%%%%%%%%

\begin{document}

  \JMUTitle
      {Seeing Through Data: A Systematic Literature Review on the Role of Visualization and Interface Design in the Organizational Use of Data Objects}        % Titel der Arbeit
      {Linus Schärmann}                        % Vor- und Nachname des Autors
      {2910412}
      
      {Wirtschaftswissenschaftlichen Fakultät}  % Name der Fakultaet
      {W"urzburg 2026}                          % Ort und Jahr der Erstellung
      {31.01.2026}                              % Tag der Abgabe
      {Tim Thorwart-Gumpert}               % Name des Erstgutachters
      {}                          % Name des Zweitgutachters

  \clearpage

\lhead{}
\pagenumbering{Roman} 
    \setcounter{page}{1}

\tableofcontents
\clearpage

%%%%%%%%%%%%%%%%%%%%%%%%%%%%
%%  Kurzzusammenfassung   %%
%%%%%%%%%%%%%%%%%%%%%%%%%%%%
\newpage
\lhead{Abstract}
\section*{Abstract}
\addcontentsline{toc}{section}{Abstract}

\newpage
\lhead{List of Figures} % Bei englischsprachiger Arbeit anzupassen auf: List of Figures
\addcontentsline{toc}{section}{List of Figures} % Bei englischsprachiger Arbeit anzupassen auf: List of Figures
\listoffigures

\newpage
\lhead{List of Tables} % Bei englischsprachiger Arbeit anzupassen auf: List of Tables
\addcontentsline{toc}{section}{List of Tables} % Bei englischsprachiger Arbeit anzupassen auf: List of Tables
\listoftables
\newpage

\setlength{\parskip}{0.5em} 


%%%%%%%%%%%%%%%%%%%%%%%%%%%%%%%%%%
%%  Definition der Abkürzungen  %%
%%%%%%%%%%%%%%%%%%%%%%%%%%%%%%%%%%
\lhead{List of Abbreviations} % Bei englischsprachiger Arbeit anzupassen auf: List of Abbreviations
\section*{List of Abbreviations} % Bei englischsprachiger Arbeit anzupassen auf: List of Abbreviations
\addcontentsline{toc}{section}{List of Abbreviations} % Bei englischsprachiger Arbeit anzupassen auf: List of Abbreviations

\begin{acronym}
  \acro{DO}{Data Object}
  \acro{ML}{Machine Learning}
\end{acronym}

%%%%%%%%%%%%%%%%%%%%%%%%%%%%
%%  Einstellungen  %%
%%%%%%%%%%%%%%%%%%%%%%%%%%%%
\clearpage
\pagenumbering{arabic}  
    \setcounter{page}{1}
\lhead{\nouppercase{\leftmark}}

%%%%%%%%%%%%%%%%%%%%%%%%%%%%
%%  Hauptteil  %%
%%%%%%%%%%%%%%%%%%%%%%%%%%%%

\section{Introduction} \label{introduction}

\citet{RN1}

\section{Theoretical Background} \label{theoretical-background}

\subsection{Boundary Objects} \label{boundary-objects}

\subsection{Data Objects as Epistemic Artifacts} \label{data-objects-as-epistemic-artifacts}

\section{Methodology} \label{methodology}

The review follows a structured approach based on the guidelines provided by \citet{RN2}. 

\subsection{Purpose} \label{purpose}

Planning - Explicit + Comprehensive + Reproducible

Purpose of this review is to identify and analyze existing literature that address the role of visualization and interface design in the organizational use of \acp{DO} within \ac{ML} contexts. 

\subsection{Draft Protocol} \label{draft-protocol}

Planning - Explicit + Comprehensive

\subsection{Practical Screening} \label{practical-screening}

Selection - Quantitative + Qualitative

\begin{itemize}
  \item Content: Data Objects, Visualization, Interface Design in Organizational Use
  \item Language: English, (German)
  \item Time Frame: 2005 - present
  \item Publication Type: Peer-Reviewed Journals, Conference Proceedings
  \item Setting: Organizational/professional
  \item Participants: Professionals, managers, analysts, data scientists
  \item Artifact: Dashboards, ML interfaces, analytics tools
  \item Design: Empirical or strong conceptual
\end{itemize}

\subsection{Literature Search} \label{literature-search}

Selection - Quantitative + Qualitative

\begin{quote}
\raggedright
\begin{ttfamily}
("data object*" OR "data artefact*" OR "boundary object*" OR "digital twin" OR "data representation") AND ("machine learning" OR "artificial intelligence" OR "algorithmic system" OR "data-driven") AND ("organization" OR "management" OR "decision making" OR "organizational knowledge") AND ("visualization" OR "interface design" OR "dashboard" OR "representation" OR "human-computer interaction")
\end{ttfamily}
\end{quote}

\subsection{Data Extraction} \label{data-extraction}

Selection - Quantitative + Qualitative

\subsection{Quality Appraisal} \label{quality-appraisal}

Selection - Quantitative + Qualitative + Explicit

\subsection{Study Synthesization} \label{study-synthesization}

Execution -  Quantitative + Qualitative + Quantitative \& Qualitative - Explicit 


\subsection{Writing the Review} \label{writing-the-review}

Execution - Explicit + Reproducible

\section{Structured Literature Review Results} \label{slr-results}

\subsection{Overview of Identified Literature} \label{overview-of-identified-literature}

\subsection{Technical Origin of Data Objects in ML Contexts} \label{technical-origin}

\subsection{Data Objects as Boundary Objects in Organizational Contexts} \label{organizational-contexts}

\subsection{Visualization and Interface Design as Epistemic Factors} \label{visualization}

\section{Discussion} \label{discussion}

\section{Future Research Directions} \label{future-research-directions}

\section{Conclusion} \label{conclusion}

%%%%%%%%%%%%%%%%%%%%%%%%%%%%
%% Literaturverzeichnis wird 
%% automatisch eingefügt
%%%%%%%%%%%%%%%%%%%%%%%%%%%%
\clearpage
\lhead{}
\printbibliography
\addcontentsline{toc}{section}{\bibname}


%%%%%%%%%%%%%%%%%%%%%%%%%%%%
%% Anhang (optional) 
%%%%%%%%%%%%%%%%%%%%%%%%%%%%
%\clearpage
%\appendix
%\section{Appendix A} % Bei englischsprachiger Arbeit anzupassen auf: Appendix A

%%%%%%%%%%%%%%%%%%%%%%%%%%%%
%% Eidesstattliche Erklärung
%% muss angepasst werden 
%% in Erklaerung.tex
%%%%%%%%%%%%%%%%%%%%%%%%%%%%
\input{Erklaerung.tex}

\end{document}